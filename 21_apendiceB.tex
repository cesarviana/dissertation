\chapter{Requisitos do Aplicativo}
\label{apendice_b}
Os requisitos são histórias de usuário conforme o modelo:
	\textit{Eu, como [papel], (preciso|gostaria) [funcionalidade] para [benefício]}.
A vantagem deste modelo é que ele mostra claramente quem precisa da funcionalidade e qual a justificativa. 

\begin{itemize}
\item RF01 – Eu, como professor, preciso ver a lista de brinquedos disponíveis para conexão sem fio, para conectar o celular ao brinquedo.
\item RF02 – Eu, como professor, gostaria que o sistema não listasse dispositivos além dos que são brinquedos RoPE, para facilitar a seleção.
\item RF03 – Eu, como professor, gostaria que o celular retornasse para a tela inicial caso a conexão seja perdida, para que seja possível reconectar.
\item RF04 – Eu, como criança, preciso ver projetado no chão uma seta indicando o resultado de cada pressionamento nos botões do RoPE, para que eu entenda o efeito de cada pressionamento.
\item RF05 – Eu, como criança, preciso que os comandos projetados sejam apagados do tapete quando forem eliminados da memória do brinquedo, para manter a sincronia de informações.
\item RF06 – Eu, como pesquisador, gostaria que de registrar o início e o fim de uma atividade, para captar as interações ocorridas durante um experimento. 
\item RF07 – Eu, como pesquisador, preciso registrar a idade e o sexo dos participantes de uma atividade, para que extrair informações demográficas de um experimento.
\item RF08 – Eu, como pesquisador, gostaria de desativar a projeção das setas sobre o tapete, para que seja possível observar reação da criança quando essa assistência for eliminada.

\end{itemize}

Já os requisitos não funcionais não estão relacionados a serviços, mas sim a propriedades de segurança, confiabilidade e tempo de resposta. Além disso, os requisitos não funcionais podem afetar a arquitetura geral de um sistema.

\begin{itemize}
    \item RNF1 – Eu, como pesquisador, preciso que o sistema funcione em dispositivos com Android 6.0, para suportar ao menos 70\% dos dispositivos Android.
    \item RNF2 – Eu, como professor, preciso instalar o sistema, para que ele abra automaticamente em tela cheia e não mostre barra de navegador.
    \item RNF3 – Eu, como criança, preciso que a luminosidade do projetor seja suficiente para eu visualizar a projeção em ambientes claros a uma distância de 1 metro entre a lente e a superfície de projeção, para possibilitar o uso em salas de aula.
\end{itemize}