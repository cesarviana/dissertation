\chapter{Requisitos do Aplicativo}
\label{apendice_requisitos}
Os requisitos são histórias de usuário conforme o modelo:
	\textit{Eu, como [papel], (preciso|gostaria) [funcionalidade] para [benefício]}.
A vantagem deste modelo é que ele mostra claramente quem precisa da funcionalidade e qual a justificativa. 

\begin{itemize}
\item RF01 – Eu, como professor, gostaria que o sistema não listasse dispositivos além dos que são brinquedos RoPE, para facilitar a seleção.
\item RF02 – Eu, como professor, gostaria que o celular retornasse para a tela inicial caso a conexão seja perdida, para que seja possível reconectar.
\item RF03 – Eu, como criança, preciso que a sequência de blocos de papelão seja transferida ao RoPE como um algoritmo, para que possa programá-lo sem tocar no celular.
\item RF04 – Eu, como criança, preciso ver projetado no chão uma marcação indicando quais blocos fazem parte do algoritmo que estou construindo, para saber quais blocos serão executados e quais não serão executados.
\item RF05 – Eu, como criança, preciso que os comandos executados pelo brinquedo sejam destacados, para que eu saiba qual bloco corresponde a qual movimento do brinquedo.
\item RF06 – Eu, como pesquisador, preciso que os programas criados pelas crianças estejam disponíveis para consulta posterior, para que sirva como insumo de análise. 
\item RF07 - Eu, como professor/pesquisador, gostaria de ativar a execução passo a passo (depuração), para a criança perceber o efeito de cada comando.
\end{itemize}

Já os requisitos não funcionais não estão relacionados a serviços, mas sim a propriedades de segurança, confiabilidade e tempo de resposta. Além disso, os requisitos não funcionais podem afetar a arquitetura geral de um sistema.

\begin{itemize}
    \item RNF1 – Eu, como pesquisador, preciso que o sistema funcione em dispositivos com Android 6.0, para suportar ao menos 70\% dos dispositivos Android.
    \item RNF2 – Eu, como criança, preciso que a luminosidade do projetor seja suficiente para eu visualizar a projeção em ambientes claros a uma distância de 1 metro entre a lente e a superfície de projeção, para possibilitar o uso em salas de aula.
    \item RNF3 - Eu, como criança, preciso que o tempo de detecção e destaque dos blocos seja menor que 500ms, para não prejudicar a usabilidade. 
\end{itemize}