\begin{agradecimentos}

    Ao meus pais. Sem vocês eu nada seria. 

    Ao meu irmão, companheiro e exemplo. Admiro muito teu trabalho e humildade. Você também foi fundamental neste trabalho. 

    Ao meu sobrinho Pedro, pelas opiniões sinceras, e também à minha irmã.

    Ao professor André Raabe, pelas portas abertas, e por plantar muitas sementes de Educação. Que elas se expandam e transformem as salas de aula em locais de construção de conhecimento.

    À Renate Raabe, pela oportunidade de participar no projeto Lite Is Cool. Este foi meu verdadeiro estágio de docência, experiência que vou levar pra vida. Aproveito e agradeço as estudantes, e ao meu colega Ivan. Creio que formamos uma bela equipe.

    Ao meu amigo Noschang, que todos chamam de mago, pelos seus poderes computacionais. Obrigado pelos ensinamentos.

    Ao Paulo, por compartilhar conhecimento e apoiar nas questões de hardware do brinquedo RoPE.

    À Maraysa, secretária do MCA, por enviar o meu resumo para tradução quando eu esqueci, e por sempre incentivar e apoiar os estudantes.

    À Sylvana, diretora do CDI, por todo apoio. Você foi fundamental para esse trabalho. Agradeço também à Secretaria Municipal de Educação de Gaspar. 

    Às professoras do CDI, pelas contribuições à pesquisa. Só posso dizer que gostaria de ser uma de suas crianças, brincando e aprendendo nos espaços que vocês constroem.
    
    Às crianças participantes. Espero que tenham se divertido. Vocês foram a parte mais importante desse trabalho.

    À Aline, minha companheira. Obrigado pelo apoio, pela paciência, e por entender os momentos de ausência. Com metade da sua agilidade essa dissertação estaria pronta em 1 ano.

    Também ao Flick, meu companheiro, muitas vezes dormindo no meu colo enquanto eu escrevia esse texto.

    Àos muitos professores que tive.

    Àos integrantes do Lite e do GIE.

    E, por fim, à CAPES e à Univali, pela bolsa de pesquisa. 

\end{agradecimentos}