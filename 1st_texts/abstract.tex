\begin{abstract}
 \begin{otherlanguage*}{english}
 
Debugging algorithms is a practice present in learning to program. The identification and correction of errors allows the student to understand the problem and their own way of thinking. For this reason, debugging algorithms was among the design principles of LOGO, one of the first environments developed for children to learn to program. Programmable toys descend from LOGO and have the same aim of allowing children to have their first contact with algorithms, but offering intuitive and playful programming interfaces. We divided these interfaces into two groups: tangible and virtual. Tangibles favor collaboration and contact with materials, and virtual ones provide powerful graphic effects. Isolated, however, these two types of interfaces are not conducive to debugging. This work aims is to explore how the union of tangible and virtual interfaces can facilitate the debugging of algorithms by children. For that, it presents RoPE AR, an augmented reality (AR) interface for the RoPE programmable toy. RoPE AR’s proposal is to use a smartphone camera to capture algorithms created with cardboard blocks; send these algorithms to RoPE to execute; and control a projector that emits virtual elements on the blocks and a projected mat. A mapping of 56 programmable toys identified that the approach suggested in this research is new, as only 2 toys use AR and none apply projection. The RoPE AR evaluation took place at a child development center with the participation of 20 children who interacted in pairs and trios, creating and debugging algorithms for RoPE to capture a projected apple. We filmed the interactions and interviewed the teachers after the activities. The interface proved workable for classroom environments, and children could create, edit, and correct errors in algorithms. However, it is not possible to state that the use of augmented reality favored debugging. Still, the experiment reveals that children perceived augmented reality, but there was no clear distinction between tangible elements and projected virtual objects. In addition, they also perceived the sound and visual feedback emitted when the toy “collides” with the virtual object. In the interviews, the teachers suggested future research could support the configuration of virtual mats to allow to explore open subjects and storytelling, going further than programming challenges with pre-defined objectives.

\end{otherlanguage*}
\end{abstract}
