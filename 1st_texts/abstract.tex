\begin{abstract}
 \begin{otherlanguage*}{english}
 
Programmable toys facilitate children’s first contacts with algorithms by offering a playful appearance and simplified interface. The main programmable toy interfaces are virtual or tangible, and each one has advantages and disadvantages. Virtuality makes it easier to change colors and highlight elements, and tangibility favors sensory stimuli and collaboration. Combining these two characteristics is a challenge. This work aims to combine the advantages of virtual and tangible interaction in an augmented reality interface for the RoPE toy. An industrial mapping show that from 56 toys, only 2 have augmented interfaces, showing the benefits from this approach is an open question. In the proposed interface, the child creates algorithms using cardboard blocks augmented by virtual elements emitted by a projector. We plan to test in case studies how this user experience affects the complexity of the programmed algorithms, the number of errors identified and the engagement observed. An online service will store the algorithms created and facilitate the extraction of information. The current results show the feasibility of using projection in artificially lit environments, but not outdoors. The next steps include designing the cardboard blocks and observing and describing children’s interactions when programming algorithms. We believe this project can contribute to the field of design of programmable toy interfaces, generating reflections on the applicability of projective and tangible augmented reality.

\end{otherlanguage*}
\end{abstract}

