\begin{abstract}
 \begin{otherlanguage*}{english}
 
Debugging algorithms is considered a necessary practice for learning programming, and it was also present in the LOGO language. Programmable toys descend from LOGO and have the same goal of allowing children to have their first contact with algorithms, but offering more intuitive and playful programming interfaces. These interfaces are divided into two groups: tangible and virtual. While tangibles favor collaboration and direct contact with materials, the virtual ones provide graphic effects that transcend the physical world. In isolation, however, these two types of interfaces do not allow debugging. The objective of this work is to investigate, in an exploratory way, how the union of tangible and virtual interfaces can facilitate the understanding and debugging of algorithms by children. For that it presents RoPE AR, an augmented reality interface for the RoPE programmable toy. The proposal of RoPE AR is to allow the creation of algorithms with cardboard blocks augmented by projected virtual elements. A mapping of 56 programmable toys identified only two that use augmented reality, and none use projective AR. The evaluation of RoPE AR took place in a child development center with the participation of 20 children who interacted in pairs and trios, solving programming problems and finding errors in algorithms. The interactions were filmed and recorded on a platform for creating and applying tests based on puzzles. Using the interface proved feasible in classroom environments, and children were able to collaboratively create, edit, and correct errors in algorithms. The children could not clearly distinguish tangible elements and projected virtual objects. As future work, it is suggested that the interface supports the configuration of virtual mats with open themes, and not just to complete programming challenges with pre-defined objectives.

\end{otherlanguage*}
\end{abstract}
