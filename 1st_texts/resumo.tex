% Formatação de Resumo - pg 267 do arquivo "univali_custom.sty"
\setlength{\absparsep}{14pt} 
\begin{resumo}
%introdução

Depurar algoritmos é uma prática presente no aprendizado de programação. A identificação e a correção de erros possibilitam que o aluno compreenda o problema e a sua própria forma de pensar.  Por isso, a depuração de algoritmos esteve entre os princípios de design do LOGO, um dos primeiros ambientes desenvolvidos para crianças aprenderem programar.  Brinquedos programáveis descendem do LOGO e tem o mesmo objetivo, de permitir que crianças tenham os primeiros contatos com algoritmos, mas oferecendo interfaces de programação intuitivas e lúdicas. Essas interfaces se dividem em dois grupos: tangíveis e virtuais. As tangíveis favorecem a colaboração e contato com materiais, e as virtuais proporcionam efeitos gráficos poderosos. Isolados, porém, esses dois tipos de interface não favorecem a depuração. O objetivo deste trabalho é explorar como a união de interfaces tangíveis e virtuais pode facilitar a depuração de algoritmos por crianças. Para isso apresenta a RoPE AR, uma interface de realidade aumentada (RA) para o brinquedo programável RoPE. A proposta da RoPE AR é usar a câmera de um smartphone para captar algoritmos criados com blocos de papelão; enviar esses algoritmos para o RoPE executar; e controlar um projetor que emite elementos virtuais sobre os blocos e um tapete projetado. Um mapeamento de 56 brinquedos programáveis identificou que a abordagem sugerida nesta pesquisa não é utilizada, pois apenas 2 brinquedos usam RA e nenhum aplica projeção.  A avaliação da RoPE AR ocorreu em um centro de desenvolvimento infantil com a participação de 20 crianças, que interagiram em duplas e trios criando e depurando algoritmos para o RoPE capturar uma maçã projetada. As interações foram filmadas e as professoras foram entrevistadas após as atividades. A interface se mostrou viável para ambientes de sala de aula, e as crianças conseguiram criar, editar, e corrigir erros em algoritmos. Entretanto, não é possível afirmar que o uso da realidade aumentada favoreceu a depuração. Ainda assim, o experimento revela que as crianças perceberam a realidade aumentada, mas não distinguiram claramente os elementos tangíveis e objetos virtuais projetados. Além disso, o feedback sonoro e visual emitido quando o brinquedo “colide” no objeto virtual parece estimular a conclusão da tarefa. Nas entrevistas, os professores sugeriram que pesquisas futuras possam apoiar a configuração de tapetes virtuais para permitir a exploração de temas abertos e de contação de histórias, indo além de desafios de programação com objetivos pré-definidos. 

\end{resumo}