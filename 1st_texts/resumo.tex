% Formatação de Resumo - pg 267 do arquivo "univali_custom.sty"
\setlength{\absparsep}{14pt} 
\begin{resumo}
%introdução
Brinquedos programáveis foram criados para facilitar os primeiros contatos de crianças com algoritmos ao oferecer aparência lúdica e interface simplificada. Os principais tipos de interface desses brinquedos são virtuais ou tangíveis, e cada tipo tem vantagens e desvantagens.
%problema
A virtualidade flexibiliza mudar cores e destacar elementos, e a tangibilidade favorece estímulos sensoriais e colaboração. Unir essas duas características é um desafio para os projetistas.
%objetivos
Este trabalho pretende unir as vantagens da interação virtual e tangível em uma interface de realidade aumentada para o brinquedo programável RoPE, um robô educacional criado para crianças a partir dos 4 anos. Um mapeamento industrial, que observa brinquedos desenvolvidos por empresas além de projetos acadêmicos, indicou que esse tipo de abordagem não é utilizada, e portanto há uma lacuna a ser investigada. A proposta de interface é que a criança crie algoritmos usando blocos de papelão aumentados por elementos virtuais emitidos por um projetor.
%métodos
Pretende-se avaliar como essa experiência de usuário afeta a os algoritmos programados, a colaboração entre crianças, e erros de localização e direção. Um serviço online foi projetado para armazenar os algoritmos criados e facilitar a extração dessas informações.
%resultados
Os resultados obtidos até o momento indicam a viabilidade do uso de projeção em ambientes iluminados artificialmente, porém não em ambientes externos. As próximas etapas incluem projetar os blocos, e observar e descrever as interações de crianças durante a programação de algoritmos. Acredita-se que esse projeto pode contribuir com o campo de design de interfaces de brinquedos programáveis gerando reflexões sobre a aplicabilidade de realidade aumentada projetiva e tangível.

% Quali e quanti paralelamente MISTA
% Estudos de Caso
% Conversas com crianças
% Teste T pareado. Botões, Intervenção, Botões
% Teste não paramétrico paramétrico
% Krushkal Walis

\begin{comment}
O Resumo é um dos componentes mais importantes do trabalho. É partir dele que o leitor irá decidir se vale a pena continuar lendo o trabalho ou não. O resumo deve ser escrito como um parágrafo único, sem utilizar referências bibliográficas e evitando ao máximo, o uso de siglas/abreviações. O resumo deve conter entre 200 e 400 palavras, sendo composto das seguintes partes (organização lógica): introdução, objetivos, justificativa, metodologia, resultados esperados ou obtidos. Esta é a seqüência lógica, não devendo ser utilizados títulos e subtítulos. Não abuse na contextualização, pois o foco deve ser nos objetivos, resultados esperados e resultados obtidos. Escreva o resumo apenas após a conclusão do trabalho. Ele deve refletir bem aquilo que foi desenvolvido.
\end{comment}

\end{resumo}