% Formatação de Resumo - pg 267 do arquivo "univali_custom.sty"
\setlength{\absparsep}{14pt} 
\begin{resumo}
%introdução

Depurar algoritmos é considerada uma prática necessária ao aprendizado de programação, e que esteve presente também na linguagem LOGO.  Brinquedos programáveis descendem do LOGO e tem o mesmo objetivo de permitir que crianças tenham os primeiros contatos com algoritmos, mas oferecendo interfaces de programação mais intuitivas e lúdicas. Essas interfaces se dividem em dois grupos: tangíveis e virtuais. Enquanto as tangíveis favorecem a colaboração e contato direto com materiais, as virtuais proporcionam efeitos gráficos que transcendem o mundo físico. De modo isolado, porém, esses dois tipos de interfaces de BPs não favorecem a depuração. O objetivo deste trabalho é investigar, de modo exploratório, como a união de interfaces tangíveis e virtuais pode facilitar a compreensão e depuração de algoritmos por crianças. Para isso apresenta a RoPE AR, uma interface de realidade aumentada para o brinquedo programável RoPE. A proposta da RoPE AR é permitir criar algoritmos com blocos de papelão aumentados por elementos virtuais projetados. Um mapeamento de 56 brinquedos programáveis identificou apenas dois que usam realidade aumentada, e nenhum usa RA projetiva.  A avaliação da RoPE AR ocorreu em um centro de desenvolvimento infantil com a participação de 20 crianças que interagiram em duplas e trios solucionando problemas de programação e encontrando erros em algoritmos. As interações foram filmadas e registradas em uma plataforma para criação e aplicação de testes baseados em puzzles. O uso da interface se mostrou viável em ambientes de sala de aula, e as crianças conseguiram criar, editar, e corrigir erros em algoritmos colaborativamente. As crianças não conseguiram distinguir claramente os elementos tangíveis e os objetos virtuais projetados. Como trabalho futuro é sugerido que a interface suporte a configuração de tapetes virtuais com temáticas abertas, e não apenas para completar desafios de programação com objetivos pré-definidos.

%problema

%objetivos
%Este trabalho pretende unir as vantagens da interação virtual e tangível em uma interface de realidade aumentada para o brinquedo programável RoPE, um robô educacional criado para crianças a partir dos 4 anos. Um mapeamento industrial, que observa brinquedos desenvolvidos por empresas além de projetos acadêmicos, indicou que esse tipo de abordagem não é utilizada, e portanto há uma lacuna a ser investigada. A proposta da interface é que a criança crie algoritmos usando blocos de papelão aumentados por elementos virtuais emitidos por um projetor.
%métodos

%resultados
%As próximas etapas incluem projetar os blocos, e observar e descrever as interações de crianças durante a programação de algoritmos. Acredita-se que esse projeto pode contribuir com o campo de design de interfaces de brinquedos programáveis gerando reflexões sobre a aplicabilidade de realidade aumentada projetiva e tangível.

% Quali e quanti paralelamente MISTA
% Estudos de Caso
% Conversas com crianças
% Teste T pareado. Botões, Intervenção, Botões
% Teste não paramétrico paramétrico
% Krushkal Walis

\begin{comment}
O Resumo é um dos componentes mais importantes do trabalho. É partir dele que o leitor irá decidir se vale a pena continuar lendo o trabalho ou não. O resumo deve ser escrito como um parágrafo único, sem utilizar referências bibliográficas e evitando ao máximo, o uso de siglas/abreviações. O resumo deve conter entre 200 e 400 palavras, sendo composto das seguintes partes (organização lógica): introdução, objetivos, justificativa, metodologia, resultados esperados ou obtidos. Esta é a seqüência lógica, não devendo ser utilizados títulos e subtítulos. Não abuse na contextualização, pois o foco deve ser nos objetivos, resultados esperados e resultados obtidos. Escreva o resumo apenas após a conclusão do trabalho. Ele deve refletir bem aquilo que foi desenvolvido.
\end{comment}

\end{resumo}