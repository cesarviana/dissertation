%% Como usar o pacote acronym


% Na primeira vez que for citado o acronimo, o nome completo 
% irá aparecer seguido do acronimo entre parênteses. Na 
% proxima vez somente o acronimo irá aparecer. Se usou a 
% opção footnote no pacote, entao o nome por extenso irá 
% aparecer no rodapé \ac{acronimo}


% Para aparecer com nome completo + acronimo
% \acf{acronimo}

% Para aparecer somente o acronimo
% \acs{acronimo}

% Nome por extenso somente, sem o acronimo
% \acl{acronimo}

% igual o \ac mas deixando no plural com S (ingles)
% \acp{acronimo}

% \acfp{acronimo}

% \acsp{acronimo}

% \aclp{acronimo}

%% ATENCAO
% Criei o comando \acfe{}, resultando em: Extenso -- ACRO

\chapter*{Lista de Abreviaturas}%
% \addcontentsline{toc}{chapter}{Lista de abreviaturas}
\markboth{Lista de abreviaturas}{}


\begin{acronym}

%A
\acro{API}{\textit{Application Programming Interface}}
%B
\acro{BLE}{\textit{Bluetooth Low Energy}}
\acro{BPs}{Brinquedos Programáveis}
%C
\acro{CDI}{Centro de Desenvolvimento Infantil}
\acro{CSTA}{\textit{Computer Science Teachers Association}}
\acro{CHI}{\textit{Human Factors in Computing Systems}}
\acro{CIEB}{Centro de Inovação para a Educação Brasileira}
%I
\acro{IBGE}{Instituto Brasileiro de Geografia e Estatística}
\acro{IDEB}{Índice de Desenvolvimento da Educação Básica}
\acro{IHC}{Interação Humano Computador}
\acro{ISTE}{\textit{International Society for Technology in Education}}
%J
\acro{JSON}{\textit{Javascript Object Notation}}
%P
\acro{PC}{Pensamento Computacional}
\acro{PPP}{Projeto Político Pedagógico}
%Q
\acro{QP}{Questão de pesquisa}
\acro{QE}{Questão exploratória}
\acro{CAQDAS}{\textit{Computer Assisted Qualitative Data Analysis Software} - Software de Apoio à Análise de Dados Qualitativos}
%R
\acro{RA}{Realidade Aumentada}
\acro{RSL}{Revisão Sistemática da Literatura}
\acro{RoPE}{Robô Programável Educacional}
%Z
\acro{ZDP}{Zona de Desenvolvimento Proximal}

\end{acronym}