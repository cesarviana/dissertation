\chapter{Resultados}
\label{c_resultados}

\section{Comparação com os trabalhos relacionados}
\label{sec:comparacao}

\section{Considerações}
\label{sec:}

%Este capítulo pode ter uma última seção como esta denominada ``Considerações'' ou ``Discussão'' consolidando a análise dos resultados.


















































































\begin{comment}

    %A força motriz do design iterativo são as metas do usuário, que são ações que o usuário deseja poder fazer \cite{rogers_design_2013}. Neste trabalho, o alcance das metas de usabilidade se dá quando a criança:
    
    \begin{itemize}
        \item posiciona o brinquedo na posição inicial;
        \item percebe que os blocos representam ações do robô;
        \item sequencia blocos formando um algoritmo;
        \item inicia execução do algoritmo;
        \item altera sequência de blocos; e
        \item percebe blocos destacados durante execução.
    \end{itemize}
    
    Para direcionar a observação, as categorias de análise serão 
    (i) para quais elementos da interface as crianças olham;
    (ii) se e como as crianças manipulam os blocos; 
    (iii) perguntas realizadas; 
    (iv) se e como as crianças comparam os blocos com os símbolos do robô; 
    (v) como ocorre o início da execução; 
    (vi) em que local e direção posicionam o robô; e 
    (vii) se e como os elementos virtuais são percebidos pelas crianças.
    
    %A colaboração será observada quando duas ou mais crianças brincarem em conjunto com os blocos. Esse tipo de evento já foi observado em estudos anteriores \cite{sapounidis_tangible_2019, raabe_estudo_2019}, mas precisa ser confirmado neste estudo para afirmar que a interface apresentada tem os benefícios das interfaces tangíveis. 
    
    % \cite{bardin_alise_1979}.
    \end{comment}