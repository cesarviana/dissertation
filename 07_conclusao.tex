% ---
% Conclusão
% ---
\chapter{Conclusão}
\label{c_conclusao}

%Quais problemas este trabalho trata
A questão que norteia esta pesquisa é como unir as vantagens das interfaces virtuais e tangíveis para facilitar o aprendizado de programação por crianças. A busca por experiências de interação benéficas ao aprendizado inicia desde a graduação do autor, na qual foi desenvolvido um aplicativo para programar e ver o algoritmo da memória no RoPE. Entretanto, os testes daquela etapa e estudos subsequentes indicaram que telas não são o tipo de interface mais adequado a crianças pequenas, as quais preferiram usar os botões físicos do brinquedo.

Como alternativa ao smartphone, foram investigadas interfaces tangíveis de programação em blocos. As interfaces tangíveis ofereceram vantagens como a possibilidade de interação coletiva e intuitiva. Nessa etapa porém, percebeu-se a dificuldade das crianças sequenciarem blocos para formar um algoritmo. Essa dificuldade se deu tanto pelo encaixe invertido dos blocos, quanto pela aparente incoerência na sequência de encaixe.

Na tentativa de contribuir para o campo de design de interação, que estuda este tipo de problema, o \autoref{c_fundamentacao_teorica} trouxe a visão dos teóricos que iniciaram o estudo com foco no desenvolvimento da criança. Piaget observou o desenvolvimento das estruturas cognitivas, e afirma que há operações mentais não realizáveis sem esses estruturas. Papert avançou neste estudo e observou como materiais externos contribuem para a formação das estruturas mentais, incentivando a abordagem por projetos significativos. Bruner teoriza sobre as formas de representação do conhecimento, e concorda que as experiências sensoriais concretas devem embasar o raciocínio formal subsequente.

O Pensamento Computacional foi abordado em seus quatro pilares \cite{brackmann_desenvolvimento_2017}: abstração, decomposição, reconhecimento de padrões e algoritmos. Após isso, foram analisados conceitos ligados à criação de interfaces de programação para crianças: brinquedos programáveis, interfaces tangíveis, programação em blocos e marcas fiduciais.

O estudo sobre o estado da arte se dividiu em três partes: pesquisas de interfaces com crianças, mapeamento industrial e trabalhos relacionados. O resultado mais relevante percebido é que há uma grande variedade de brinquedos programáveis disponíveis no mercado, porém o uso de realidade aumentada nesses brinquedos é escasso. 
O objetivo 1 \textit{(mapear os principais tipos de interfaces de brinquedos programáveis existentes)} foi alcançado nessa etapa. Não foi encontrado outro mapeamento abrangente com foco nessas interfaces. A análise dos resultados indicou aumento do número de brinquedos lançados a partir de 2013, quando também aumentou o interesse pelo assunto \acl{PC}. Os tipos de interfaces mais usados são virtuais, e a realidade aumentada é pouco explorada.

O segundo objetivo \textit{(implementar um sistema de registro de interações com o brinquedo RoPE)} é considerado atendido por meio da implementação de uma \ac{API}\footnote{\url{https://github.com/cesarviana/rope_api}}. Em pesquisas anteriores, o autor precisou anotar manualmente ações ocorridas com o brinquedo. Esse tipo de coleta é propensa a erros de anotação e demanda tempo. O método automatizado pode ser útil para outras pesquisas. 

O terceiro objetivo \textit{(construir uma interface de realidade aumentada projetiva para o brinquedo programável RoPE)} foi atendido parcialmente. A estrutura da interface (projeção, captura de blocos, identificação da área projetável e projeção de elementos virtuais na posição desejada) foi implementada. Um mini-projetor de 1200 lumens produziu imagens visíveis em ambientes internos iluminados. Os blocos com marcas fiduciais com diâmetro de 3cm foram identificados sem problemas por uma câmera com resolução 8 megapixels de resolução. Porém essa parte técnica ainda não é a interface. Ainda é preciso implementar a comunicação Bluetooth entre o brinquedo e o smartphone, e avaliar o uso por crianças.

Por fim, o quarto objetivo \textit{(planejar e aplicar um protocolo de experimentos para coletar eventos de interação com a interface)} teve apenas o planejamento executado. O \autoref{c_avaliacao} descreve a intenção de se utilizar a abordagem centrada no usuário, o que inclui design iterativo e observações empíricas, bem como os aspectos definidos a priori para serem observados. O planejamento da atividade para detectar dificuldade com localização e direções foi apresentado como uma sequência de 9 atividades, em que a criança programa o brinquedo em orientações variadas. 

%O que foi feito até o momento (cap 2, 3 e 4)

%Este trabalho

%Quais os próximos passos (com base no cap 4)

%Resultados esperados... no modelo Univali está assim:

%Este capítulo deve apresentar uma síntese sobre o trabalho desenvolvido, realizando uma análise a respeito do cumprimento dos objetivos estabelecidos e da verificação da hipótese de pesquisa inicial. Cada objetivo deve ser analisado, identificando-se o grau de atendimento (parcial ou integral), os problemas encontrados e as soluções adotadas, e justificando o porquê do não cumprimento integral (quando for o caso). Não devem ser apresentadas justificativas baseadas em dificuldades de natureza pessoal (ex. falta de tempo).


%_________________________
\section{Contribuição da Dissertação}
\label{c_conclusao-contribuicao}

%Nesta seção, devem ser destacadas as principais contribuições do trabalho. Deve se identificar a relevância técnico-científica da pesquisa realizada, assim como os seus impactos social, ambiental e econômico (quando aplicável). Principalmente, deve-se ressaltar a contribuição do trabalho em relação ao estado da arte. Também podem ser identificados resultados alcançados quanto à publicações e patentes depositadas.

A idealização de uma ferramenta de representação externa de realidade aumentada é uma das contribuições deste trabalho até o momento. Não se encontrou nenhum brinquedo que utilize essa abordagem, que pode representar uma inovação.

Também se considera uma contribuição a implementação do sistema de registro de interações. Esse sistema foi desenvolvido utilizando ferramentas comumente utilizadas no mercado de trabalho, o que facilita encontrar desenvolvedores que possam manter e evoluir o sistema. Deste modo, o sistema poderá ser utilizado em pesquisas futuras.

O mapeamento de 56 brinquedos programáveis em uma planilha pública\footnote{url{https://bit.ly/35cUbZ1}} também é um subproduto que poderá ser útil para pesquisas futuras. O maior número de brinquedos analisados em pesquisas \cite{hamilton_emerging_2020,yu_review_2019} foi 30, e portanto os resultados obtidos são mais abrangentes. Além dos 56 brinquedos, há uma lista de brinquedos encontrados que não foram analisados, mas que constituem o corpus de resultados. O fato da planilha ser pública e manter o histórico de alterações permite adição de novos registros sem risco de perder a informação atual.

%_________________________
\section{Trabalhos Futuros}
\label{c_conclusao-trabalhos-futuros}

%Esta seção deve identificar possíveis trabalhos que possam ser realizados a partir do desdobramento da pesquisa feita na dissertação. Procure discutir esses trabalhos como oportunidades de pesquisa que possam ser aproveitadas tanto por você como por outras pessoas.

%Caso queira listar essas oportunidades, anteceda a lista por um parágrafo introdutório, como, por exemplo: “Ao longo do desenvolvimento deste trabalho, puderam ser identificadas algumas possibilidades de melhoria e de continuação a partir de futuras pesquisas, as quais incluem:”. Depois do parágrafo inicial, você pode listar as melhorias e continuações que podem ser feitas a partir do trabalho desenvolvido, mas procure comentar um pouco sobre cada proposta, mostrando que você já saberia como começar aquela nova pesquisa.

A principal continuação deste trabalho será executar o design iterativo e as avaliações descritas no \autoref{c_avaliacao} para responder as hipóteses de pesquisa. Para isso é necessário executar o plano de avaliação descrito no \autoref{c_avaliacao}.

Foram vislumbradas algumas possibilidades de investigação futuras a esta pesquisa. Uma é verificar como as cores dos botões influenciam na comunicação de locais e direções. Poderia se criar um brinquedo com todos os botões da mesma cor e observar como adulto e criança se comunicam durante a construção de um algoritmo. Isso poderia evidenciar a importância das cores das interfaces de programação para crianças.

Outra possibilidade também é avaliar formas alternativas de comunicar algoritmos, como a fala ou gestos por exemplo. Tendo em vista que é possível programar o RoPE por Bluetooth, um adulto escondido pode observar a criança enquanto ela fala ou gesticula e programar o brinquedo de acordo. Isso poderia gerar \textit{insights} de formas mais naturais da criança programar.

Um estudo abrangente, com mais crianças, também é um trabalho futuro útil para verificar a atração e o engajamento obtido com a interface. \citeonline{horn_comparing_2009} posicionaram uma interface em um museu e observaram variáveis como tempo de uso e colaboração dos visitantes.

Por fim, a possibilidade de criar mapas virtuais que tenham reações aos movimentos do brinquedo programável também é um trabalho a ser desenvolvido. Identificar a posição do brinquedo e modificar seu ambiente permite criar jogos e desafios dinâmicos. A aplicação a atividades de aprendizado específicas precisa ser avaliada, para indicar se vale a pena ou não prosseguir com essa abordagem de uso de \acl{RA}.