\chapter{Conclusão}
\label{c_conclusao}

O tema central desta pesquisa foi a depuração de algoritmos por crianças. A motivação para tal temática é a presença da depuração como um dos princípios de design do ambiente Logo. Esse ambiente, desenvolvido por Papert e sua equipe na década de 1980, foi uma das primeiras ferramentas na área da Informática na Educação com foco em possibilitar a crianças aprenderem programação. E mais do que aprender programação, a proposta é permitir que a criança aprenda sobre a sua própria forma de pensar. 

O ambiente Logo sofreu alterações de design desde a sua criação. A tartaruga robótica originou brinquedos programáveis, os quais possuem diversos formatos que vão desde outros animais, veículos, até personagens robóticos de madeira. Deste modo, a aparência lúdica desses brinquedos os tornou mais aderentes ao público infantil. Essa pesquisa apresentou 56 exemplos desses brinquedos, através de um mapeamento industrial sistemático. 

Esse mapeamento também apresenta outra transformação de design: novos métodos de inserir comandos para esses brinquedos/computadores executarem. O que antes as crianças faziam usando teclados, exigindo alfabetização, agora é possível por blocos de madeira, blocos eletrônicos, aplicativos de smartphone, botões acoplados, e botões de controle remoto. Essa pesquisa classificou essas interfaces em dois grupos: tangíveis e virtuais.

Argumentamos que as interfaces tangíveis, de modo isolado, não oferecem um suporte ideal para que crianças depurem algoritmos. Blocos de madeira carecem de mecanismos que promovam a visibilidade, descrita por Norman (1988), e indiquem qual ação está sendo executada em cada momento. As alternativas existentes, como o Cubetto e o Robo-Blocks necessitam de componentes eletrônicos específicos. Essa dependência não é um problema em produtos acabados, mas o processo de design de ferramentas educacionais necessita de diversas iterações e alterações, nesse caso depender de componentes eletrônicos pode atrasar o desenvolvimento. Outro aspecto, ligado à visibilidade, é seu impacto na construção do modelo mental, ou seja, o entendimento que o usuário cria sobre como um dispositivo funciona. 

As interfaces virtuais, por outro lado, também não são adequadas para crianças, principalmente quando utilizadas de modo passivo e sem supervisão (STIGLIC; VINER, 2019). Esses equipamentos também são geralmente de uso individual, o que não favorece a colaboração e interações interpessoais, fundamentais para o aprendizado.

Esse trabalho propôs a RoPE AR como uma alternativa. Ela utiliza realidade aumentada projetiva para adicionar elementos virtuais sobre elementos tangíveis. Deste modo, a criança pode criar algoritmos com blocos de papelão e os elementos virtuais destacam e aumentam o significado desses blocos. O uso de papelão e elementos gráficos facilita criar de novos designs de blocos e atividades, pois é um material acessível.

Partindo desta proposta, a presente pesquisa analisou a existência de trabalhos similares. Com isso, apresentou trabalhos que aplicam RA projetiva na medicina, na indústria e também em produtos educacionais. O trabalho de Beraza, Pina e Demo (2016) se destaca por também aplicar realidade aumentada projetiva em um tapete virtual para Bee-Bot, a qual é programada por botões. O diferencial da RoPE AR é explorar a construção de algoritmos com blocos tangíveis, e promover a interação dos blocos e do brinquedo programável com objetos virtuais.

Neste contexto, surgiu a seguinte questão de pesquisa: como crianças que são público-alvo de um brinquedo programável, poderiam interagir com uma interface de programação focada em permitir visualizar os passos e a execução de um algoritmo? Quatro objetivos específicos operacionalizam a resposta a essa questão.

O primeiro objetivo - Descrever sistemas e aplicações que utilizam realidade aumentada projetiva - foi atendido pela fundamentação teórica. O trabalho de Wrzesien et al. (2015), destinado ao tratamento de fobias, e o trabalho de Roberto (2013), uma plataforma de realidade aumentada, são exemplos citados. Esses trabalhos evidenciaram a viabilidade da técnica escolhida, e a sua aplicabilidade a ferramentas educacionais. 

O segundo objetivo - Mapear as principais categorias interfaces de brinquedos programáveis existentes - foi atendido através de um mapeamento industrial sistemático. Esse mapeamento listou os brinquedos disponíveis comercialmente, e também incluiu os brinquedos citados em revisões como as de Hamilton et al. (2020) e de Yu e Roque (2019). Esse mapeamento evidenciou que a realidade aumentada é pouco usada em brinquedos programáveis.

O terceiro objetivo — Construir a interface de realidade aumentada projetiva para o brinquedo RoPE — ocorreu com a modelagem e implementação da RoPE AR, descritas no Capítulo 4. Essa construção abrangeu alterar o firmware do RoPE para se comunicar com um dispositivo móvel; implementar essa comunicação em tal ambiente; desenhar os blocos tangíveis; implementar a projeção sobre os mesmos. Nessa etapa também foi implementada a integração com a Ct Puzzle Platform, que viabilizou configurar os ambientes projetados pela RoPE AR e também armazenar as interações em registros para posterior processamento.

Por fim, a interface possibilitou atender ao objetivo quarto - Experimentar a interface em atividades com crianças e coletar eventos de interação. Essa experimentação ocorreu em um CDI, com a participação de 20 crianças. Elas programaram e corrigiram algoritmos, em duplas e trios, com auxílio do pesquisador e também de professoras. Vídeos das atividades e entrevistas com as professoras formam os dados captados nesta etapa.

Após o cumprimento destes objetivos, por fim, o Capítulo 6 tenta responder à questão de pesquisa, de como as crianças interagiram com tal ambiente. Para isso, analisa vídeos das interações e entrevistas com as professoras de modo indutivo e dedutivo. As conclusões, não definitivas, são as seguintes:

\begin{enumerate}
    \item As crianças entendem o significado dos desenhos de cada bloco (\autoref{quadro:reconhecimentoblocos}), e tendem a falar de “esquerda” e “direita” como “pro lado”
    \item As crianças relacionam as cores dos blocos com as cores dos botões do brinquedo RoPE. Também entendem como encaixar esses blocos para formar sequências válidas, com o bloco verde na primeira posição (\autoref{fig:relacao_blocos})
    \item As crianças percebem quando elementos projetados surgem e somem do mapa
    \item As crianças estranharam o fato de um elemento virtual desaparecer, e procuram a maçã dentro da “barriga” do robô. Há forte integração entre virtualidade e realidade, o que e as crianças podem não distinguir (\autoref{quadro:percepcao_ra})
    \item As crianças reagem à função de destacar os blocos do algoritmo encaixando ainda mais blocos na sequência (\autoref{fig:sequencia_blocos}), mas isso aconteceu apenas em três momentos
    \item O engajamento na tarefa é indicado por (\autoref{fig:acoes_geral_programacao} e \autoref{fig:acoes_geral_programacao}) menos de 3\% das ações representaram estados de distração ou repouso. 
    \item O efeito sonoro e visual do brinquedo capturando a maçã projetada parece influenciar no engajamento
    \item As crianças não perceberam ou não valorizaram o destaque dos blocos em cada movimento do brinquedo. Sua atenção se voltou ao brinquedo
    \item As crianças não depuram os algoritmos por conta própria. Entretanto, as atividades focadas em depuração permitiram avaliar o algoritmo, identificar erros, localizar e corrigir bugs. Mas isso dependeu da orientação do pesquisador, e em apenas uma atividade permite afirmar que a realidade aumentada influencia a depuração, ao menos na faixa etária estudada
    \item As crianças têm forte atração pelo feedback imediato (NORMAN, 1998) fornecido pelo clique dos botões, o que as fez esquecerem de usar os blocos de papelão
    \item As crianças precisam de mais tempo nas fases de programação envolvendo giro (\autoref{fig:tempo_tarefa})
    \item As fases de depuração demandam menos tempo de resolução (\autoref{fig:tempo_tarefa}) do que as tarefas de programação, mas não foi possível afirmar qual o motivo
    \item As professoras avaliam positivamente a aplicação proposta, e sugerem atividades condizentes com a união de interfaces tangíveis e realidade aumentada. Também não temem que as crianças danifiquem os blocos de papelão
    \item A similaridade de cores entre blocos é um fator causador de erros
\end{enumerate}

\section{Contribuições da Dissertação}
\label{c_conclusao-contribuicao}

Essa pesquisa teve abordagem exploratória, e sua principal contribuição é apresentar resultados empíricos sobre uma temática que carece de ser explorada. Neste sentido, as conclusões podem servir de base para próximos trabalhos de caráter quantitativo para avaliar hipóteses aqui levantadas.

A construção e avaliação da RoPE AR é também uma contribuição. O fato de estar conectada a uma plataforma de configuração de testes permite definir outros mapas, sem a necessidade de recompilar o software. Deste modo o seu uso não fica restrito aos objetivos do presente trabalho.

O mapeamento de 56 brinquedos programáveis em uma planilha pública\footnote{\url{https://bit.ly/35cUbZ1}} também é um subproduto útil para pesquisas futuras. O maior número de brinquedos analisados em pesquisas \cite{hamilton_emerging_2020,yu_review_2019} foi 30, o que indica que os resultados obtidos na presente pesquisa são mais abrangentes. Além dos 56 brinquedos, a lista contém brinquedos encontrados que não foram analisados, e podem integrar o corpus mapeado. O fato da planilha ser pública e manter o histórico de alterações permite adição de novos registros sem risco de perder a informação atual.

A implementação da biblioteca de comunicação por Bluetooth com o RoPE via sistema Android\footnote{\url{https://github.com/cesarviana/rope_ar}} também é um subproduto que pode apoiar aplicações que necessitem comunicar ações ou obter informações do brinquedo a curta distância. 

Por fim, as entrevistas transcritas representam uma fonte de análise para futuras pesquisas. Este trabalho analisou apenas os aspectos focados na atividade, mas o questionário aberto apresenta outras observações das professoras. Um tema abordado foi a formação tecnológica das professoras de educação infantil.

%_________________________    
\section{Trabalhos Futuros}
\label{c_conclusao-trabalhos-futuros}
Foram vislumbradas algumas possibilidades de investigação futuras a esta pesquisa. Entre elas, está a construção de uma plataforma onde as professoras possam definir mapas, elementos virtuais, e sons presentes nas atividades. Seria um modo de suportar as aplicações sugeridas, como contação de histórias e jogos, e associar a criatividade das professoras ao ambiente proposto.

Ao externalizar os comandos que podem ser programados no brinquedo RoPE, não há mais a limitação de espaço físico nos botões do brinquedo. Pode-se criar blocos com finalidades específicas, como programar sons, movimentos de amplitudes variadas e programar efeitos luminosos. Também podem ser criados blocos que quando executados, causem efeitos não apenas no brinquedo, mas também no mapa, como mudar a cor do tapete ou mostrar outro cenário. Uma linha de pesquisa seria, portanto, identificar como usar esse potencial para favorecer o aprendizado em diferentes faixas etárias.

A identificação da posição do brinquedo RoPE também abre espaço para novas aplicações. Uma aplicação possível seria criar um bloco de caneta, semelhante à presente no ambiente Logo. Contudo, diferente de usar uma caneta física, a marcação da trajetória do brinquedo seria uma linha projetada. Essa linha pode ter, portanto, diferentes cores e formas.

Outra possibilidade é avaliar métodos alternativos de comunicar algoritmos, como a fala ou gestos. Tendo em vista que é possível programar o RoPE por Bluetooth, um adulto escondido pode observar a criança enquanto ela fala ou gesticula e programar o brinquedo de acordo. Isso poderia gerar \textit{insights} de formas mais naturais da criança programar.

\section{Aspectos éticos}
Essa pesquisa foi aprovada pelo Comitê de Ética em Pesquisa da Univali, conforme parecer 4.027.404. Nenhuma criança ou professora foi identificada, seja por informações pessoais ou imagens. As professoras autorizaram o compartilhamento das entrevistas, e o acesso ao CDI foi autorizado pela direção e pela Secretaria Municipal de Educação.
