[Pesq] Então, fiz umas perguntas, pois eu falei né, que pesquisa é qualitativa. Eu preciso ter uma noção geral de como é, não apenas este trabalho que desenvolvi. Então eu queria que você falasse um pouco sobre o teu trabalho, de quanto tempo trabalha com crianças, e como é a rotina.

[Prof\_4] Deixa eu te falar. A gente aqui trabalha com uma professora e com uma auxiliar. Então a professora gerente da sala é ela, eu sou só a auxiliar. Apesar de eu ser formada e pós graduada em pedagogia. Tu não tem problema disso?

[Pesq] Não.. Eu não sabia qual era a professora, e como falei contigo primeiro...

[Prof\_4] Porque ela não gosta, não entende muito de tecnologia, ela disse "fica pra ti"... porque a gente trabalha bem em conjunto assim, eu e ela.

[Pesq] Legal, também posso perguntar algo pra ela então depois..

[Prof\_4] É, não sei se ela quer responder...

[Prof\_5] Pode ser...

[Prof\_4] Disse pra ele, que a gente trabalha bem em conjunto.

[Prof\_5] É, a gente trabalha juntas, então o que ela falar tá perfeito. Tá bom pode ficar a vontade.

[Pesq] Então, é sobre isso né, como é a rotina, de como é o trabalho com as crianças.. 

[Prof\_4] Neste ano tu quer saber?

[Pesq] No geral, desde quanto tempo trabalha com crianças, qual é tua experiência...

[Prof\_4] Eu já to trabalhando com Educação Infantil já faz uns 15 anos. Vim morar pra cá, já comecei, fiz a faculdade de pedagogia, sou pós graduada em gestão, orientação e supervisão. Aí fiquei sete anos afastada da sala, fiquei na direção, fiquei como diretora da creche do Bairro\_1, não se se tu conhece. 

[Pesq] Estudei no IFSC, mas não conheço.

[Prof\_4] É bem em frente. Tem a escola e tem a creche. Eu fiquei sete anos de secretária, depois fiquei quatro de diretora, daí eu vim, fiquei de diretora da creche ali da Marinha, ali onde chamam de Marinha. Aí eu cansei, daí tava com saudade de ficar na sala, e como moro aqui pertinho vim pra cá. Voltei pra sala em 2017. Que é o que eu gosto, estar com eles, esse retorno deles que eles te deram hoje assim, eu sou apaixonada. Por mais que eles não saibam, mas como eu disse, eles acabam dando umas dicas, eles soltam umas coisas que a gente nem imagina né. As vezes a gente faz um planejamento e sai exatamente outro porque eles vem com ideia que a gente nem pensou que a gente nem imaginava que eles iam perceber, então acabam mudando assim, mas isso é o que me faz me apaixonar pela Educação Infantil assim, essa falta de maldade que eles tem né, porque eles fazem as coisas por pura inocência. Então a nossa rotina agora com a pandemia mudou um pouquinho, então a gente tá com a turma dividida. Essas crianças vem essa semana, depois ficam em casa e vem o outro grupo. Então é uma experiência nova também que a gente tá vivenciando né, devido à pandemia. Eu começo aqui no CDI às 8, aí eles tem o café da manhã, depois tem o planejamento. Vinte pras onze é o almoço deles, eles almoçam escovam o dente e começam ir embora. À tarde vem uma outra turma, daqui a pouquinho eles vão pro café, às 15 horas. E também né, nos outros dias que tu não tá aqui a gente faz o planejamento, faz outras coisas, faz bastante brincadeira. Essa idade que eles tão vai de 4 a 5 anos. Tem os que fizeram 4 anos até 31 de março, agora a partir de abril a gente tem aqueles que vão começar fazer 5 anos. Aí a gente já iniciou bastante assim, a questão de números né, mas tudo brincando né. Esse trabalho aqui eles que foram no computador, eles pintaram, eles foram colocando as quantidades, a letra dos nomes, a gente começou com a inicial. Então é uma amostra pra eles de como que é a diferenciação de números, de letras, mas nada alfabetizando, porque eles vão ter tempo bastante pra isso. Então é tudo em cima de brincadeiras, de jogos, como tu trouxe pra gente, foi bem legal. Como te disse, o olhar pedagógico ali são cores, são números ali, como eles contaram quantas vezes vai pra frente, quantas vezes vai pra trás, achei bem bacana assim, bem legal. Eles terem que pensar se vai passar da maçã, vai ter que voltar, vai ter que virar, vai ter que... acho que ali trabalha muita coisa junto, muita coisa num brinquedo só, achei bem bacana assim. Então é o que a gente trabalha aqui, é a brincadeira, o quebra cabeça... é montar pecinhas, é eles contarem o que vai precisar pra montar, o que vai precisar ter.

[Pesq] Eu percebi que tinha uns quebra-cabeças sobre a mesa, tive a impressão de que conseguiram encaixar bem as pecinhas (do projeto), apesar de não ser um formato bem certinho como no quebra-cabeça... Eles foram encaixando naturalmente.

[Prof\_4] É como eu te disse hoje de manha. Devido à pandemia, como eles ficaram em casa um ano, a gente percebeu que esse ano tá bem diferenciado. Quando eles passavam de uma turma pra outra, iam mais ou menos igual, todos sabendo a primeira letra do nome.. todos sabendo os números de 0 a 9 né, e esse ano não, esse a gente percebeu bastante diferença por eles terem ficado esse ano em casa. Então veio, como hoje de manhã a Menina\_0, a Menina\_1, que já pegaram né, a Menina\_2, já tão sabendo os números, já sabem as letras, então são crianças que talvez os pais ficaram mais em cima em casa, fizeram as atividades online... E aí já tem esses, como tu viu a tarde, que já tiveram um pouquinho mais de dificuldade né, de raciocínio, de manter a questão de quantidade, esse ano de pandemia a gente notou bastante essa diferença.

[Pesq] Tem um impacto grande na criança.

[Prof\_4] Isso. Também é entendível né, porque muitos pais saíam pra trabalhar, e as crianças ficavam com vó, não conseguiam acompanhar as atividades online que a gente mandava pra casa... e aí faz falta né essa convivência, porque acaba um ensinando o outro. "Ah, esse aqui é 2 porque tem duas coisas", mesmo que eu não digo mas tem um que já se mete, já diz e em casa não tem isso né. Então a gente esse ano tem essa missão de igualar um pouquinho eles, pra ano que vem que eles vão pro pré, que ainda é aqui, e depois eles vão pra escola. 

[Pesq] O que você pensa sobre as crianças desta idade (4 a 5) tendo contato com tecnologia. 

[Prof\_4] Ah, eu acho necessário. Eu acho necessário mas também tem que ter o equilíbrio assim. Tu não pode deixar a criança só com tecnologia, vamos supor, o celular, que é o que eles tem em mãos, né, que a gente viu que foi o que aconteceu com alguns, que passaram a pandemia toda com o celular na mão, tem que ter o equilíbrio, mas também acho que com a pandemia a tecnologia veio mais ainda né, e a gente viu até com os professores. Como eu te disse, a Prof\_5 já é uma pessoa um pouquinho mais velha, então ela tem grandes dificuldades com a tecnologia. A gente até recebeu da prefeitura uns chrome books, então eu estou dando bastante, apesar de eu não ter esse conhecimento todo, porque eu já me acho ultrapassada né, mas aí ela eu tô ajudando bastante ela nessa questão, porque o chrome book é um pouquinho diferente. Mas eu disse pra ela, depois que aprender a mexer ele é muito prático, porque fica tudo gravado no drive, do drive já consigo mandar imprimir, então achei ele bem prático. Mas de todas as professoras aqui, eu acho que só nós estamos usando o chrome book, porque as outras tem essa dificuldade. Então eu penso que se as crianças que estão nascendo, que estão nessa idade, não tiverem contato com a tecnologia, vão chegar umas pessoas mais velhas igual nós assim né, meio que analfabetos na tecnologia, e eu acho que a pandemia veio para a tecnologia ainda mais pra nossas vidas, porque tudo hoje em dia né.. A gente mesmo teve que se adaptar à aulas online, à internet, à plataforma, e aí que vai né. Eu acho que eles precisam estar por dentro disso tudo senão ficam pra trás no mercado de trabalho também.

[Pesq] Uma coisa que defendo neste trabalho, na questão dos bloquinhos, é a visibilidade. A criança poder ver os blocos, e poder ver o programa que ela tá criando. Diferente de tá inserindo o programa nos botões, e não ver esse programa criado. Uma outra criança que tá ali do lado ela também pode ver esse programa. Eu queria saber tua visão a respeito de outras atividades que envolvem essa questão da visibilidade, onde a criança tem contato com algo que ela está vendo na frente dela, o que isso impacta nas outras atividade.

[Prof\_4] Eu acho que isso entra no equilíbrio também, porque aí eles não ficam só na tecnologia, mas eles precisam vivenciar e visualizar com os bloquinhos, então não fica só no apertar o botão, como no celular, pode ver que eles apertar, apertar ali né. Então se tu não chamar a atenção ali pros bloquinhos... Então acho que é o equilíbrio ali da tecnologia, não fica só no apertar o botão, mas sim em terem que olhar os bloquinhos, ter que visualizar, ter que fazer a sequencia, pra tecnologia poder andar né. Acho que faz eles pensarem, faz terem que prestar atenção nos bloquinhos, e dá o equilíbrio da tecnologia.

[Pesq] Eu vejo essa questão por exemplo aqui nesse painel. Tem os números né. Tá uma coisa bem exposta, a criança chega e tá vendo isso na frente dela, diferente de falar "um", o que é um? 

[Prof\_4] E até foi engraçado ontem, isso aqui foi uma proposta de três dias. Primeiro eles pintaram o cartaz, que a ideia era eles escreverem os números, alguns conseguiram definir um pouquinho mais, alguns não, e aí depois eu chamei, foram só três lá no computador da secretaria, pra imprimir os números né. Eu disse "Vamos com a Prof\_4 lá", e eles "fazer o que?" "Imprimir os números no computador, vai sair numa impressora", e eles "você vai deixar nós mexer no computador?" Parece uma coisa que vocês não podem tocar. "Sim, a gente vai mexer", mostrei onde ficavam os números. Então acho que é interessante eles escreverem, pintar com lápis, com caneta, como foi no nosso tempo, mas é interessante também eles irem lá e ver que os números saem de uma maquina também né.

[Pesq] A gente olha e não percebe os detalhes... Tem um número desenhado ali.. 

[Prof\_4] Como eu disse, eu as vezes não vou perceber no teu trabalho tecnológico algumas coisas como você não vai perceber no trabalho pedagógico outras. 

[Pesq] Nessa ideia de projetar alguma coisa no chão, e ter essa interação de um robô, onde se tem a percepção de algo que chegou perto de algo, e aí acontece um evento. No caso ali é um robô que chega perto da maçã, e aí faz o som de comer a maçã. Tu imagina, alguma atividade que pudesse usar esse mesmo princípio, desta mecânica, vamos dizer assim? Uma imagem, um cenário no chão, e algumas coisas espalhadas assim que pudessem interagir.

[Prof\_4] É, tipo se fosse uma contação de história? Vai contando uma história e o robô vai demonstrando um cenário, vai chegando perto do cenário e vai fazendo um personagem da história. Eu pensaria nisso... Tipo um lobo, chegou no mato e o robô segue o lobo, algo assim, e vai chegando...

[Pesq] Por exemplo, se chegou perto do lobo, faz o barulho do lobo... 

[Pesq] Que problemas que tu imagina que este tipo de projeto pode ter no dia a dia, aqui, vamos dizer, de uma creche, no teu trabalho, que sugestões poderia dar pra melhorar o projeto. Pra melhorar em algum aspecto. 

[Prof\_4] Eu não vi problema nenhum... na minha área pedagógica... na tua área as vezes tu pode ter percebido algumas mudanças em algumas coisas, mas assim... 

[Pesq] Uma apreensão que eu tinha é, pela estrutura a criança derrubar e quebrar o projetor.

[Prof\_4] Mas aí, vamos supor que tu deixasse esse equipamento na minha mão. Antes de tudo isso eu iria explicar pra eles que isso ali é um projetor, que teriam que cuidar pra não bater, porque também a gente trabalha com projetor na parede né, então quando a gente traz a gente diz que só a prof pode mexer, eles não podem mexer. A gente tá usando o chrome book agora, então eles também tem esse contato assim. Eu não vejo como um perigo pra eles. Isso não ia cair em cima de uma criança e machucar. No máximo iria quebrar, no máximo teria um prejuízo. A gente se preocupa mais se fosse uma coisa muito pesada e poderia machucar uma criança. A gente iria trabalhar com eles o cuidado com o equipamento. Tu viu que foi o que gerou bastante curiosidade pra eles né. Porque quando eles chegaram aqui, eles questionaram o que era. E a outra professora disse que tu iria mostrar um robô e imaginaram que isso ali fosse o robô. Por isso ficaram com toda essa expectativa, porque imaginam que o robô grande... 

[Pesq] Até na sala da Prof\_3 eles fizeram um robô lá... 

[Prof\_4] Eu acho que isso aqui, daqui pra frente, vai render outros planejamentos, porque daí vão começar a perguntar, disso, vão começar lembrar.. Os pais vão vir nos questionar, porque daí eles chegam em casa e começam a conversar. E daí da fala deles, do questionamento, a gente dá uma continuidade no planejamento. 

[Pesq] Acho que, nossa, vocês me ajudaram de mais assim. Existe a parte de programação, e etc. Mas isso não vai ser nem de longe o forte do trabalho, vai ser o que eu peguei aqui, o que aconteceu. Vão me perguntar como as crianças reagiram.

[Prof\_4] Mas eles são muito imprevisíveis, não são?

[Pesq] Pois é, e as vezes a resposta que parece óbvia falam algo totalmente diferente.

[Prof\_4] Por isso estava sentido contato assim, na direção você trabalha mais com os adultos, a prefeitura, os pais, os funcionários, então chegou num ponto que o adulto me encheu o saco assim, aí quando voltei pra sala assim tem muitas vezes que tenho que sair, virar e rir, porque não pode dar aquela confiança todas, aí a gente ri assim, depois volta... Mas é bem apaixonante..

[Pesq] Agradeço a ajuda.

[Prof\_4] A gente achou bem bacana também. A gente abre as portas aí. A professora liberando, se tu precisar voltar.. estamos aí.
[fim]