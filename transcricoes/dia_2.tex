[Pesq] São umas perguntas bem gerais, só para eu conhecer do trabalho, porque a pesquisa é qualitativa, então eu tenho que entender não só sobre o projeto, mas também sobre o contexto geral. Primeiro eu gostaria de perguntar se posso gravar isso porque depois é melhor..

[Prof\_3] É melhor pra transcrever né, pq se a gente for anotando perde muita coisa né.

[Pesq] Eu me acostumei muito a digitar, então só pra escrever (no papel)... Então. Eu gostaria que você falasse sobre o teu trabalho.. Quanto tempo tem contato com as crianças, eu percebi que é bastante tempo né, que tu trabalha com crianças.

[Prof\_3] Sim, aham. Bom eu já trabalho com crianças desde 1992. Comecei só na Educação Infantil, e aí depois na Secretaria da Educação, com a Educação Infantil, coordenando todas as... e aí a gente começou o processo de fazer uma proposta pedagógica, que até então não tinha, então a gente conduziu, fez formação, tudo pra montar a proposta né. E depois eu saí da Secretaria, voltei pra sala de aula. Desde 2014 então eu estou em sala de aula, com as várias turmas. Então a gente planeja,.. a gente usava a proposta pedagógica, que é as Linguagens, Brincadeiras e Interações, e depois quando teve agora a BNCC a gente tá seguindo os campos de experiências, os 5 campos de experiências.

[Pesq] Quando que entrou em vigor essa parte da BNCC.

[Prof\_3] Oficialmente foi ano passado, então a gente teve que seguir a proposta, porque a gente usou bastante documentos do MEC também pra elaborar né. Então ela é próxima à BNCC, só que ela trabalhava mais com os conceitos né, e a BNCC trabalha mais com os campos de experiência amplos e com os objetivos que ali tem né... por faixa etária né...

[Pesq] A BNCC não fala como tem que ser feita a atividade...

[Prof\_3] Não, ela tem objetivos e a gente, como professora, tem a liberdade de escolher qual a atividade que mais próxima ou mais no contexto né. Então, como a gente aqui construiu os cantos, todos esses cantos que tem aqui foram ideias das crianças né. A BNCC fala dos espaços como forma educativa também né. Então a gente pega das crianças pra ampliar, então todos esses espaços são sugestões que eles deram. Então aqui no robô, que eles queriam um robô. E aí eu disse "mas como a gente vai fazer um robô" e eles "ah, a gente vai fazer com um controle remoto", e disse "mas a gente não tem essa tecnologia, como a gente podia fazer?". Aí um disse "ah, coloca fios, tem que colocar fios" e eu disse "mas a prof não entende de elétrica".

[Pesq] Eu lembro que ontem eu tirei a extensão da mochila e o menino falou "olha quanto fio!".

[Prof\_3] Aham. Então eles entendem que há uma coisa de funcionamento né, que tem que ter fios, que tem que ter... mas aí a gente construiu então de sucatas, pra guardar os carrinhos, porque esse canto aqui eles gostam muito que tem os carrinhos né, tem as pistas que a gente ganhou de doação, então eles amam isso. A gente fez pra organizar mais né. Então ele não tem uma funcionalidade assim, remota né.

[Pesq] Mas ele já entendem que...

[Prof\_3] Sim, eles já entendem que existe uma tecnologia. Então eu trouxe pra eles também um vídeo sobre os robôs né.. eles assistiram, então a gente vai ampliando, traz do mundo deles pra ampliar um pouco mais.

[Pesq] Isso era um filme, eu sei que tem aquele filme dos robôs.

[Prof\_3] Não, é mais tecnológico assim, a gente trouxe robôs de verdade.

[Pesq] Tu já falou um pouco sobre isso né, mas o que tu pensa sobre as crianças tendo contato com tecnologia em geral?

[Prof\_3] Eu acho fantástico, é uma pena que nós não temos né. Na rede pública é mais difícil porque ter mais equipamentos né, de pesquisa com as crianças, como uma mesa digital, isso é muito custoso né, pro município ter. A gente sabe que nem temos espaço pra isso, tem que ter uma sala especial porque os materiais são caros e não dá pra ficar aqui e ali, se perdem e não tem contato. Mas isso é interessante trazer né, porque outra pessoa que tenha o entendimento isso eu achei muito legal assim, porque eles gostaram assim, eles ficaram vidrados naquele robozinho. "E como é que comeu a maçã" eu achei engraçado né ele olhou por baixo onde é que vai a maçã né. Mas eu acho que ainda é um pouco distante. Talvez né daqui mais algum tempo, quando isso ficar mais acessível. Porque eu sei que um laboratório, informática e tudo, digital, isso é um investimento alto né, então fica mais nas universidades. Eu sei que na Furb, a gente foi lá ver essa mesa digital. É muito legal. Porque nós professores, nós não temos a formação tecnológica né. Então pra gente lidar com o computador... Agora o município nos disponibilizou um computador. Ele é um chrome book. Só que nós não sabemos lidar. Eu fiquei no meu computador mesmo, porque ele não tem o pacote windows. Então, como é que a gente, que mal sabe mexer naquele, vai aprender a mexer com esse tipo.. Que vai ter que armazenar em outro tipo de formato... aí como vou fazer os vídeos, porque a gente faz vídeo, a gente faz vídeo, a gente elabora várias coisas né. Eu já fazia antes, sempre usei um pouco de vídeo com as crianças né, filmava e tudo mais. Mas agora com a pandemia, foi muito mais necessário tá disponibilizando isso né. Porque quando eles estavam em "home-office", a gente fazia vídeos pra explicar o que era a brincadeira. Pra eles nos verem, pra lembrar de nós. Então, tudo isso pra, isso tudo é difícil pra gente pegar esse fio da meada, de montar né...

[Pesq] É complexo né...

[Prof\_3] É difícil, e demora porque quando você vai editar né. Porque quando você vai editar... A edição leva muito tempo.. então cortar.

[Pesq] Então eles deram um computador pra cada professora?

[Prof\_3] Pra cada professora.

[Pesq] Mas não deram formação pra crianção de...

[Prof\_3] Sim, prometeram que vão dar uma formação. Então eu só não devolvi o notebook pra eles, porque a gente assina um termo que se acontecer alguma coisa nós temos que pagar. Então, eu vou esperar a formação, pra ver se eu consigo... se não for vantajoso eu vou devolver. Então esse tipo de coisa é bastante complicado né. Então eu que mais ou menos né.. sou meio burrinha na questão de... gosto muito de tecnologia, uso bastante, mas ainda não tenho toda... imagina quem tem mais dificuldade. Porque teve professoras que tinham muita dificuldade, então isso é complicado né. Então eu acho que na formação do professor hoje em dia, na formação universitária, tem que ter a formação de tecnologia. Tem que ter na grade, porque isso é imprescindível. A gente viu agora né, nesses tempos, que é importante a gente ter esse tipo de formação na própria formação do professor inicial. 

[Pesq] Uma coisa que você falou né, de ter um laboratório de informática e tal né.. eu li um artigo, sobre o RoPE, em que ai idéia é inverter essa lógica do laboratório de informática. Vamos dizer, uma escola, tem 45, 50 minutos de aula. O professor ele quer fazer uma atividade no laboratório de informática, até todo mundo sentar, ir e voltar meia hora de aula perdida. E aí ela fala que a ideia do RoPE é inverter, trazer a informática pra dentro da sala. Claro que esse negócio que eu fiz não é tão portátil assim, mas o robô em si ele é pequeno. Podia ter três, quatro aqui e as crianças brincando. Claro que não é "oh", um computador né que a criança faz tudo.

[Prof\_3] Mas é legal isso mesmo, tirar um pouco essa questão de ficar na máquina sentado e olhando pra algo. Ter um concreto no caso, ele é tipo, ele é 4D, ele tá ali junto com as crianças. (Diferente de) que ficar clicando ali na setinha né. Isso é interessante, pros menores ia ser interessante também. E pros maiores seria como o mecanismo que isso acontece né.

[Pesq] Por dentro ali né?

[Prof\_3] Ahan.

[Pesq] Na verdade até a gente aprende com o robô lá na universidade. Eu tô aprendendo, to mexendo dentro dele ali, essa semana deu um erro que não conseguia gravar mais nada nele e tive que ir lá em balneário, pedir ajuda pro Eng\_0, que foi o cara que projetou, e ele me mostrou um monte de coisa, como fazia né, porque já tinham tido esse mesmo erro. Uma vez ficaram um mês, vi eles batendo cabeça e daí ele resolveu. Eu não tenho um mês. Eu também to aprendendo, só que um nível mais pra dentro do robô.

[Prof\_3] O legal né, como eu tava falando pra você né. Pros menores é interessante eles seguirem um circuito completo, como eu falei pra você né. Eles vão seguir aquele roteiro pra poder encontrar, pra poder fazer todo o trajeto. É melhor pro entendimento deles. E aí então, pros pequenos... Porque no caso você tá fazendo essa pesquisa com todos... só a partir de quatro anos.

[Pesq] Eu to testando, na verdade, a ideia de quatro a seis. Mas é testar essa abordagem né. Talvez pra quatro não seja interessante, ou será que ajuda a criança mais velha entender, ou mais nova... 

[Prof\_3] Ou trazer várias facetas, na mesma atividade vários níveis.

[Pesq] No caso o robô em si ele é programado pelos botões originalmente, e tem a teoria de design que fala que as coisas tem que estar visíveis. E o que eu vejo é que nem sempre o programa está visível por meio dos botões, principalmente pra programas mais longos. Então a criança vai lá, apertou vários botõezinhos, então ela esquece aquilo, ou ela não consegue alterar. Então a ideia dos blocos é permitir que isso se torne mais visível. O programa, é um algoritmo que a gente fala, é uma sequência de passos. A única diferença é que ela tá visível por meio dos blocos, tanto a estrutura, que é a sequência de um bloquinho depois do outro, quanto a execução, que seria "esse bloco que está executado agora".

[Prof\_3] Ah tá, porque mais fácil de fazer a leitura. Aquela leitura que coloca quando você junta a programaçãozinha aqui embaixo ele já lê e já executa né.

[Pesq] Isso, pra criança conseguir perceber né. Aquelas marquinhas redondinhas, o ideal é que nem tivesse aquilo. Eu tive que usar por questão de tecnologia, pra conseguir implementar né. Então a ideia é a criança conseguir entender que o algoritmo é aquela sequência de blocos, e qual que está sendo executado.

[Prof\_3] Mas eu acho que foi bem legal assim, mesmo tendo assim alguns que são menores, mas eu acho que deu pra eles estarem fazendo né, eu acho que foram bem assim (risos).

[Pesq] Eu também achei. O Menino\_M achei que não ia conseguir e conseguiu.

[Prof\_3] Conseguiu? Ah que bom, que ele tem bastante dificuldade de entendimento assim. Mas que bom.

[Pesq] Tem mais duas perguntas aqui. A importância que tu dá pra questão da visibilidade, das coisas estarem visíveis pra criança conseguir pegar na mão, o quão importante isso é pra criança entender...

[Prof\_3] Ah, sim. Pros menores, até a criança de sete, oito anos, é sempre concreto né, é sempre a partir daquele material, que ele possa pegar, visualizar, pra ele internalizar. Porque o abstrato assim, ficar explicando muito, eles não vão conseguir fazer a associação. Eles precisam associar que uma coisa é ligada a outra. Então eles tem pegar, eles tem que sentir, eles tem que ver. Então essa forma ali eles conseguiram ter os comandos na hora. E no computador ele vai ter só uma setinha, que ele vai né, e vai colocar né, tipo clicar. E aqui não, ele vai ter que fazer algo, teve que construir algo e entender aquilo pra depois programar né. 

[Pesq] Eu vi que tem uns pedacinhos de madeira ali né, perguntei pra Prof\_6, sobre o quela aquilo né, ela falou que é sobre o contato com a natureza.

[Prof\_3] Isso, é um outro elemento. A gente está acostumado com plástico. Então plástico, madeira, são outras possibilidades de construção. Esses dias ele tava fazendo um prédio, disse que tava fazendo um prédio. Ele empilhou todos os toquinhos assim. Então é isso mesmo, é construir ou construir uma cerca. O bloco é essa tentativa. Ele não é um objeto muito estruturado. A criança tem que ter contato com objetos não estruturados. Então ele não tem forma, ele não deve pra um devido fim. Ele vai montar uma estrutura a partir daquele objeto que não tem formato. Ele que vai ter que imaginar e dar forma aquele objeto. Então o objeto não estruturado é imprescindível que tenha no espaço de educação. 

[Pesq] Um objeto que não foi decidido que...

[Prof\_3] É, ele não tem uma função. Então, "ah, isso aqui é um jogo" então tu vai completar assim. Ele é um material que dá pra fazer mil coisas.

[Pesq] Algo aberto...

[Prof\_3] Isso, aberto. 

[Pesq] Então no caso essa atividade que trouxe tá um pouco estruturada né.

[Prof\_3] É estruturada, tem objetivos e regras. Tens que montar primeiro, ver a sequência ali, então pra ele estrutura isso... por isso que os menores não tem essa... eles vão lá, vão brincar né, os menores. Os maiores vão entender que tem uma regra, se só pode apertar "esse botão", se "só pode seguir esse trajeto", "não pode fazer diferente". Então ele tem uma estrutura, e um formato a seguir. E outros objetos, como esse ali (gravetos de madeira) não, você vai ter que criar e você vai ter que fazer. Então ele vai ter contato com tudo isso. Eles vão começar a perceber né. E nessa questão das regras, de jogos, dessas atividades que tem objetivo, eles vão aos poucos entendendo. Tem coisas que posso fazer assim, tenho a liberdade de criar, e tem coisas que não, tenho que pensar, tenho que seguir certas regrinhas pra chegar no objetivo. Até conseguir alcançar.

[Pesq] Eu estou fazendo um trabalho voltado pra crianças, mas não tenho muito contato com criança né. Tenho um sobrinho de 8 anos...

[Prof\_3] É porque quem tá na academia, quem lida com as outras áreas, mesmo professor "de área né", ele não vai ter essa didática que a gente tem diferente. Ele vai tentar ver o desenvolvimento infantil, como que eles vão aprender, de que forma... Mas isso é interessante porque vocês também contribuem conosco né, porque a gente não tem essa visão que vocês tem né. De elaboração, de estruturação de pensamento. Assim, criar este tipo de jogo. Então é uma coisa muito legal. Então a gente vai ver se isso tem coerência ou não né. Aplicando com as crianças. Alguma coisa a gente pode achar que vai dar certo e as vezes não gosta. Mesma coisa a gente. As vezes a gente planeja e eles não gostam. Então não foi "aquilo tudo". Há momentos em que acontece isso também com a gente. A mesma coisa é o desenvolvimento de um jogo, depois tu vai testar pra ver se realmente pega esse público, se este público vai conseguir, pra poder aprimorar né, é bem isso mesmo.

[Pesq] Aqui eu cheguei, na outra sala, e funcionou tudo bem. Aqui já tive bem mais dificuldade.

[Prof\_3] Essa sala aqui eu não sei, se é porque ela fica bem no meio e pega bem a banda das duas, e sempre não consegue (conectar no wi-fi). É bem difícil. Até o datashow também é.

[Pesq] Vocês usam? (datashow).

[Prof\_3] Sim, mas agora ele está com uns probleminhas técnicos. O mouse né, ele não tem o plug que você pode comprar. Você vai ter que levar na assistência técnica pra poder soldar porque ele é interno. O fiozinho é interno. Não dá pra gente usar. 

[Pesq] Talvez usar televisão.. tem TV aqui..

[Prof\_3] É, essa é antiguinha, mas a mais moderna fica ali fora.

[Pesq] Nessa ideia de projetar as coisas no chão e ter uma interação 

[Prof\_3] isso é legal, isso é muito bom, gostei bastante. Eu achei que você ia colocar ali, como a gente tá acostumado né. Mas no chão foi muito legal...

[Pesq] E tu imaginas alguma atividade, que você falou né sobre não ser estruturado, onde as crianças pudessem inventar alguma atividade que usasse essa ideia de projeção, e de ah, perceber que uma coisa chegou perto da outra e fazer um som. Tu imagina alguma atividade que desse pra fazer, usando isso?

[Prof\_3] Usando esse... essa tecnologia de projetar e fazer uma brincadeira. 

[Pesq] Isso.

[Prof\_3] Ah, legal... Poderia ser... tem que pensar né... Poderia até fazer esse jogos de tabuleiros, jogos de tabuleiro. Dependendo onde você vai tem que avançar na casa, retroceder, voltar no jogo... que é interessante também né pros menores, porque eles gostam bastante. Tem jogos mais simples, jogos mais complexos. Tipo um caça palavras ali, pras crianças que são alfabetizadas seria interessante. E daí eles vão descobrir uma letra, uma palavra, ou a figura e ter que descobrir... igual eu fiz com eles ontem, tipo uma caça ao tesouro. A gente escondeu os dinossauros e aí eu gravei num aplicativo o monstrinho falando que escondeu e tinha pistas. Aí tinha várias perguntas. Então pode desenvolver um jogo que tenha essas perguntas e eles vão ter que adivinhar onde está e voltar no jogo. Então você faz um jogo aqui parado e um jogo iterativo procurando coisas. "Em certo local", e dá as características do local pra criança pensar, qual é esse local que tem essas características? Ah, tá lá, lá está a pista. Pega a pista e volta no tabuleiro. Faz uma jogada de várias coisas. Aí tem mais outra pista, em outro local. Vai ter que decifrar um enigma... então na verdade dá pra usar essa mesma tecnologia nesse sentido né. Ou montar as peças do próprio robô, pros mais avançados. Cada local tem uma parte da peça que vai dar início pra continuar o jogo.

[Pesq] Algo que não fica só parado.

[Prof\_3] Não, fica dinâmico. Você vai ter que movimentar a escola toda. E pode outras turmas participarem, de vários enigmas, de várias coisas. Pode ser a escola inteira no mesmo jogo. Então cada sala vai ter uma tarefa diferente. E no final soma os pontos pra ver quem foi a turma que teve mais pontos que acertou. E pode ter vários temas. Por exemplo. O nosso projeto de instituição é sobre música. É ligado à música na Educação Infantil, e são várias possibilidades. Esse é o nosso tema. Então a gente faz atividades voltado a isso. Então se a escola tem esse tipo de jogo iterativo, que possa incluir, cada escola possa agregar, pode desenvolver e movimentar a escola toda naquela jogo ali. Um único jogo, várias participações. Pra vocês que vão pensar na tecnologia a ser desenvolvida...

[Pesq] Pois é, muitas vezes a gente sabe como fazer, mas "o que" fazer é o mais difícil. Faltam ideias.

[Prof\_3] Mas tem vezes que parece que o cérebro secou, não tem mais nada. Eu acho que assim são os escritores né. As vezes eu tô inspirada, como eu vi agora um outro jogo bem legal que a gente vai fazer também, de perguntas e respostas num boliche. A gente vai usar não um boliche, mas outros tubos. Foi uma ideia que eu vi e pensei que a gente pode adaptar pro que a gente tá estudando sobre os dinossauros e aí a gente vai poder perguntar e saber um pouco deles. Cada vez que eles vão derrubando, cai e a pergunta é colocada pras equipes. Foi uma inspiração. Ainda bem que encontrei isso, foi um outro jogo, de outra coisa, mas que me inspirou a fazer. Assim a gente vai, a gente se inspira em alguma coisa e "ah, ainda bem que veio isso". Senão a gente fica sem ideias.

[Pesq] Um teórico que a gente estuda, o pai da informática na educação, o Seymour Papert, fala sobre as ideias poderosas. Tem toda um sequencia sobre o que é uma ideia poderosa. Por exemplo, a escrita. É algo que te ajuda a pensar. Não tá ligada a uma cultura específica, serve pra todas as culturas. Serve pra construir coisas, inventar coisas. São blocos de palavras que vai montando ao infinito de textos, de livros. Tem mais uns critérios... O LEGO por exemplo, permite criar coisas infinitas. E da mesma forma esta ideia que tu falou, é uma atividade que podes utilizar pra qualquer tema, qualquer pergunta. A programação. Assim como a escrita, tu consegue criar infinito algoritmos.

[Prof\_3] Isso aí acho tão... a gente que é leiga, olha assim, nossa, aquelas várias sequencias, aquelas coisas... Fica assim "nossa".

[Pesq] A gente tem essa ideia de que é difícil... Eu entrei num curso de informática por acaso. Quando eu ia jogar no computador, desligava o computador porque não sabia fechar a tela. As vezes não é tão complicado. Querendo ou não, o que a gente fez hoje foram algoritmos.

[Prof\_3] Olha, interessante.

[Pesq] É uma sequência de passos, pra chegar num objetivo. A nossa vida são muitos algoritmos. Ah, vou botar uma roupa. A gente segue um algoritmo que é uma sequência de passos pra gente conseguir resolver um problema. Você pensando na tua aula, tem uma sequencia de passos pra resolver. As vezes, a gente pensar um pouco melhor, ah, posso fazer isso antes, ou isso depois, ou duas coisas ao mesmo tempo. Minha mãe faz isso sem ninguém ter ensinado, ela estudou até a terceira série. Ela bota uma coisa pra começar, e vai fazer outra porque consegue fazer mais de uma coisa ao mesmo tempo. Isso são algoritmos, a única diferença é a forma como a gente escreve, que não é no computador.

[Pesq] Alguma ideia que passaram na minha cabeça, sobre aplicabilidade deste projeto. Gostaria de ver se achas que tem lógica ou não. Outros blocos que daria pra colocar, pro brinquedo rope poderia agir diferente. Um deles eu pensei, seria um bloco de caneta. A criança colocaria o bloco, e a partir daquele bloco, onde ele andasse iria fazer um risco. Seria o baixar a caneta. Como se tivesse uma caneta dentro dele. E teria outro bloco de erguer a caneta, que a partir dali ele iria parar de desenhar. Seria algo sem um objetivo dado, a criança poderia desenhar o que quisesse. 

[Prof\_3] Ela é que vai riscar...

[Pesq] Teria um bloquinho de caneta. Teria um desenho que seria uma caneta abaixada. A partir daqui ele baixaria a caneta, mas ele não tem uma caneta de verdade. 

[Prof\_3] Ele vai marcar, mas virtualmente, isso? Aí vai ficar o rastro do trajeto que ele fez? Ah, interessante também. 

[Pesq] Isso serviria pra desenhar qualquer coisa. Poderia ter um bloco pra dizer quando ele vai girar, em vez de girar 90 graus. Dizer, ah, ele vai girar mais que 90, então ele em vez de desenhar um quadrado, ele vai desenhar um triângulo por exemplo. Outros blocos que imagino. Poderia ter blocos de som, talvez. Ele chegou ali, ele faz um som, que a criança pode gravar talvez.

[Prof\_3] Ou ele fez um som e aparece... vamos supor, se for um instrumento musical, se aparece um som daquilo ele pode estar identificando, por exemplo, deu um som de trovão, tem a figura do trovão e ele vai ter que associar. É um outro jogo no mesmo (estilo), mas com outra função, que eu posso associar. Porque quem tá fazendo associação de sons, né, discriminação auditiva, brincadeiras de discriminação auditiva, ele pode chegar em determinado local, fazer um som, e daquilo a criança vai ter identificar qual é pra seguir adiante. É um outro objetivo dentro desse que tu falou. Ou até uma letra. Ele chega até uma letra, ele faz o som. Ou até um número. Pode ter vários tipos. Perguntas e respostas. Você pode usar pra vários fins. Então você pode, ter várias funções, pra ajudar o professor na sala. Como você falou, em vez de eu sair pra ir na sala de informática, vem pra auxiliar naquilo que o professor tá dando. Porque as vezes é difícil você fazer a multidisciplinaridade que falam. Mesclar os temas. Porque os conceitos são juntos, a gente que separa para fins didáticos. E você fica naquilo. Uma coisa não tem conexão com a outra. E aí num jogo assim ele vai ter a conexão de todas, de ciência com matemática... As coisas não são dissociadas, a gente dissocia e o cérebro fica todo embananado, são coisa juntas né. Então essa seria uma questão que vocês poderiam estudar neste sentido, de ajudar o professor a fixar os conteúdos com essas perguntinhas, que essa coisa que o robozinho vai... Porque não é só comer a maçã, ele tem que comer a maçã e cumprir certas coisas. E pode associar, além da brincadeira, ele vai ter conhecimento. Algo que não é massante, ele vai brincar, que vai ser legal, que vai descobrir. Porque criança gosta muito de enigmas. Associar isso a enigmas pra descobrir, pra ir... Ele vai fazer a marquinha da caneta, e com isso ele vai mudando de sequência, pode mudar de cor. 

[Pesq] Obrigado pela ajuda, vai ser muito útil...

[Prof\_3] Ah, que bom. Então certo. Se precisar de alguma coisa...
[fim