[Pesq] Vocês poderiam falar um pouco sobre o trabalho de vocês, quantos anos são professoras, como é a atividade aqui, pra que eu entenda um pouco e possa conhecer.

[Prof\_1] Uhum. Ah... 
    
[Pesq] É informal.

[Prof\_1] Tá. Eu vou falando e você vai me perguntando. 
    
[Pesq] Uhum, sim.

[Prof\_1]  Eu trabalho na rede a 8 anos, estou no 8º ano, sou formada em pedagogia, tenho pós graduação em Educação Anos Iniciais e Educação Infantil, e.. essa turma aqui é dos 5 a 6 anos, é a primeira vez que pego essa turma aqui mas eu já trabalhei com todas as turmas, todas as idades. Eu gosto muito, eu me identifico muito, eu me identifico com a turma do pré. Porque daí é o letramento...
    
[Pesq] Então já começa trabalhar com o alfabeto, isso?

[Prof\_1] É.. Não é o alfabeto, é uma introdução né, a alfabetização que a gente chama é só lá na escola
    
[Pesq] Só pode conhecer as letras assim...

[Prof\_1] Pode, reconhecer, escrever o nome... cores.
    
[Pesq] Você já tem experiência a bastante tempo né, o que você pensa sobre as crianças usando tecnologia?

[Prof\_1] Eu acho bom, muito interessante. No meu TCC eu fiz ludicidade na Educação Infantil, então a tecnologia é bem importante. É interessante introduzir isso. 
    
[Pesq] E como você acha que ela pode contribuir ou não contribuir pro desenvolvimento da criança?

[Prof\_1] Eu penso que sim, porque é uma forma educativa de inserir eles na informatização, porque em casa o celular e a tecnologia é só voltada pra YouTube, coisas que não são tão úteis. E se a gente puder introduzir isso de uma forma que eles procurem no Google o conhecimento, informática.. isso é bem legal. Interessante. Mudar o foco do YouTube...
    
[Pesq] E qual a importância que você daria para as crianças estarem vendo as coisas. Por exemplo, se eu vou tentar explicar alguma coisa só falando será que isso é suficiente ou será que eu mostrar pra criança uma imagem.. de que modo isso influencia no aprendizado dela. 

[Prof\_1] Ah, eu acho importantíssimo, porque eu sou visual, eu aprendo muito mais olhando e ouvindo, presencial. Tirar dúvidas né, na hora.
    
[Pesq] E na questão dos materiais, se vou tentar explicar alguma coisa só falando ou tendo algum objeto que eu possa apontar e querer explicar algo mostrando aquilo pra criança. 

[Prof\_1] Ah sim, é a ludicidade né, tem que mostrar, aprende muito mais. 
    
[Pesq] É isso que estou tentando né. O robô pode ser programado usando os botões, mas a criança não vê o que está sendo programado. Ela pode apertar várias vezes os botões e esquecer. E aí, usando os blocos, o que estou tentando trazer..

[Prof\_1] É na prática né..
    
[Pesq] .. é que ela tá vendo..

[Prof\_1] Não, isso é fundamental, eu acho bem interessante.
    
[Pesq] Pensando neste projeto específico, o que você achou ele? A respeito da aplicabilidade, se fosse pra ser usado aqui.. ou em alguma outra sala de aula, será que seria usável, pra alguma atividade semelhante, usando essa ideia de projetar a imagem e ter um robozinho se mexendo...

[Prof\_1] Eu acho bem legal, porque é convidativo para eles assim. Isso poderia ser usado como se a gente colocasse um reto-projetor pra eles?
    
[Pesq] Sim, a diferença é...

[Prof\_1] Tu tá encontrando ali pelo celular.
    
[Pesq] Tá no celular, só que a criança não interage com o celular, ela interage com o celular.

[Prof\_1] E com a imagem..
    
[Pesq] E com a imagem. Quando o robô chega naquela posição o celular entende e daí ele dá um feedback, faz um som do robô comendo a maçã. Mas isso poderia ser evoluído, por exemplo, passou em cima de uma ponte, faz o barulho da ponte, ou chegou no número 3, fala "chegou no numero 3", ou se tem um obstáculo faz o robô parar, porque não pode passar ali.

[Prof\_1] É como um jogo né.
    
[Pesq] Uhum.

[Prof\_1] É um jogo... bem legal, de comandos né.
    
[Pesq] E alguma sugestão que você daria que pudesse melhorar o projeto, ou a ideia. ou alguma coisa que você já viu, que pudesse ser mais útil pra vocês e pro desenvolvimento da criança. Teria alguma sugestão, "ah muda isso".

[Prof\_1] Ah, não sei dizer, isso é novo pra mim... eu nunca tinha visto isso...
    
[Pesq] Estou tentando trazer uma proposta nova...

[Prof\_1] Já tinha visto Prof\_2? 

[Prof\_2] Não, isso é totalmente novo pra gente. Isso é totalmente novo, acredito que daqui mais um tempo vai ser super necessário e a gente torce pra que isso realmente acontece porque hoje a tecnologia está influenciando todas as áreas e acho que a educação tanto infantil como anos iniciais.. tem que ser tudo tecnológico hoje.
    
[Pesq] Eu imagino pro desenvolvimento depois de adulto..

[Prof\_2] É, deixa te contar uma história. Estou na faculdade também, estou no último período de pedagogia. E quando eu comecei fazer os estágios eu penei muito, até hoje. Eu vou fazer um curso, nem que seja básico de informática, pra estar, ali pelo menos pra conhecer algumas coisas.
    
[Pesq] Edição de...

[Prof\_2] Exatamente... Então eu acho isso muito importante, tem que ter inserido já bem antes
    
[Pesq] E o desenvolvimento do raciocínio lógico também.. das direções... A parte que eu desenvolvi foi essa ideia da projeção e da comunicação com o celular. Mas o robô já existem a bastante tempo, e é usado em outras creches em Balneário e Itajaí, só o robô. As crianças adoram.

[Prof\_2] Imagina.
    
[Pesq] A minha parte foi uma parte pequena no projeto, e estou testando essa nova abordagem, isso que estou tentando avaliar. Minha parte foi só um pedacinho.

[Prof\_2] Mas que não deixa de ser importante né.